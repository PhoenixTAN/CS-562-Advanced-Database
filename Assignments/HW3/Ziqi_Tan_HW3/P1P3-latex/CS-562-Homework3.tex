\documentclass[11pt]{article}

    \usepackage[breakable]{tcolorbox}
    \usepackage{parskip} % Stop auto-indenting (to mimic markdown behaviour)
    
    \usepackage{iftex}
    \ifPDFTeX
    	\usepackage[T1]{fontenc}
    	\usepackage{mathpazo}
    \else
    	\usepackage{fontspec}
    \fi

    % Basic figure setup, for now with no caption control since it's done
    % automatically by Pandoc (which extracts ![](path) syntax from Markdown).
    \usepackage{graphicx}
    % Maintain compatibility with old templates. Remove in nbconvert 6.0
    \let\Oldincludegraphics\includegraphics
    % Ensure that by default, figures have no caption (until we provide a
    % proper Figure object with a Caption API and a way to capture that
    % in the conversion process - todo).
    \usepackage{caption}
    \DeclareCaptionFormat{nocaption}{}
    \captionsetup{format=nocaption,aboveskip=0pt,belowskip=0pt}

    \usepackage[Export]{adjustbox} % Used to constrain images to a maximum size
    \adjustboxset{max size={0.9\linewidth}{0.9\paperheight}}
    \usepackage{float}
    \floatplacement{figure}{H} % forces figures to be placed at the correct location
    \usepackage{xcolor} % Allow colors to be defined
    \usepackage{enumerate} % Needed for markdown enumerations to work
    \usepackage{geometry} % Used to adjust the document margins
    \usepackage{amsmath} % Equations
    \usepackage{amssymb} % Equations
    \usepackage{textcomp} % defines textquotesingle
    % Hack from http://tex.stackexchange.com/a/47451/13684:
    \AtBeginDocument{%
        \def\PYZsq{\textquotesingle}% Upright quotes in Pygmentized code
    }
    \usepackage{upquote} % Upright quotes for verbatim code
    \usepackage{eurosym} % defines \euro
    \usepackage[mathletters]{ucs} % Extended unicode (utf-8) support
    \usepackage{fancyvrb} % verbatim replacement that allows latex
    \usepackage{grffile} % extends the file name processing of package graphics 
                         % to support a larger range
    \makeatletter % fix for grffile with XeLaTeX
    \def\Gread@@xetex#1{%
      \IfFileExists{"\Gin@base".bb}%
      {\Gread@eps{\Gin@base.bb}}%
      {\Gread@@xetex@aux#1}%
    }
    \makeatother

    % The hyperref package gives us a pdf with properly built
    % internal navigation ('pdf bookmarks' for the table of contents,
    % internal cross-reference links, web links for URLs, etc.)
    \usepackage{hyperref}
    % The default LaTeX title has an obnoxious amount of whitespace. By default,
    % titling removes some of it. It also provides customization options.
    \usepackage{titling}
    \usepackage{longtable} % longtable support required by pandoc >1.10
    \usepackage{booktabs}  % table support for pandoc > 1.12.2
    \usepackage[inline]{enumitem} % IRkernel/repr support (it uses the enumerate* environment)
    \usepackage[normalem]{ulem} % ulem is needed to support strikethroughs (\sout)
                                % normalem makes italics be italics, not underlines
    \usepackage{mathrsfs}
    

    
    % Colors for the hyperref package
    \definecolor{urlcolor}{rgb}{0,.145,.698}
    \definecolor{linkcolor}{rgb}{.71,0.21,0.01}
    \definecolor{citecolor}{rgb}{.12,.54,.11}

    % ANSI colors
    \definecolor{ansi-black}{HTML}{3E424D}
    \definecolor{ansi-black-intense}{HTML}{282C36}
    \definecolor{ansi-red}{HTML}{E75C58}
    \definecolor{ansi-red-intense}{HTML}{B22B31}
    \definecolor{ansi-green}{HTML}{00A250}
    \definecolor{ansi-green-intense}{HTML}{007427}
    \definecolor{ansi-yellow}{HTML}{DDB62B}
    \definecolor{ansi-yellow-intense}{HTML}{B27D12}
    \definecolor{ansi-blue}{HTML}{208FFB}
    \definecolor{ansi-blue-intense}{HTML}{0065CA}
    \definecolor{ansi-magenta}{HTML}{D160C4}
    \definecolor{ansi-magenta-intense}{HTML}{A03196}
    \definecolor{ansi-cyan}{HTML}{60C6C8}
    \definecolor{ansi-cyan-intense}{HTML}{258F8F}
    \definecolor{ansi-white}{HTML}{C5C1B4}
    \definecolor{ansi-white-intense}{HTML}{A1A6B2}
    \definecolor{ansi-default-inverse-fg}{HTML}{FFFFFF}
    \definecolor{ansi-default-inverse-bg}{HTML}{000000}

    % commands and environments needed by pandoc snippets
    % extracted from the output of `pandoc -s`
    \providecommand{\tightlist}{%
      \setlength{\itemsep}{0pt}\setlength{\parskip}{0pt}}
    \DefineVerbatimEnvironment{Highlighting}{Verbatim}{commandchars=\\\{\}}
    % Add ',fontsize=\small' for more characters per line
    \newenvironment{Shaded}{}{}
    \newcommand{\KeywordTok}[1]{\textcolor[rgb]{0.00,0.44,0.13}{\textbf{{#1}}}}
    \newcommand{\DataTypeTok}[1]{\textcolor[rgb]{0.56,0.13,0.00}{{#1}}}
    \newcommand{\DecValTok}[1]{\textcolor[rgb]{0.25,0.63,0.44}{{#1}}}
    \newcommand{\BaseNTok}[1]{\textcolor[rgb]{0.25,0.63,0.44}{{#1}}}
    \newcommand{\FloatTok}[1]{\textcolor[rgb]{0.25,0.63,0.44}{{#1}}}
    \newcommand{\CharTok}[1]{\textcolor[rgb]{0.25,0.44,0.63}{{#1}}}
    \newcommand{\StringTok}[1]{\textcolor[rgb]{0.25,0.44,0.63}{{#1}}}
    \newcommand{\CommentTok}[1]{\textcolor[rgb]{0.38,0.63,0.69}{\textit{{#1}}}}
    \newcommand{\OtherTok}[1]{\textcolor[rgb]{0.00,0.44,0.13}{{#1}}}
    \newcommand{\AlertTok}[1]{\textcolor[rgb]{1.00,0.00,0.00}{\textbf{{#1}}}}
    \newcommand{\FunctionTok}[1]{\textcolor[rgb]{0.02,0.16,0.49}{{#1}}}
    \newcommand{\RegionMarkerTok}[1]{{#1}}
    \newcommand{\ErrorTok}[1]{\textcolor[rgb]{1.00,0.00,0.00}{\textbf{{#1}}}}
    \newcommand{\NormalTok}[1]{{#1}}
    
    % Additional commands for more recent versions of Pandoc
    \newcommand{\ConstantTok}[1]{\textcolor[rgb]{0.53,0.00,0.00}{{#1}}}
    \newcommand{\SpecialCharTok}[1]{\textcolor[rgb]{0.25,0.44,0.63}{{#1}}}
    \newcommand{\VerbatimStringTok}[1]{\textcolor[rgb]{0.25,0.44,0.63}{{#1}}}
    \newcommand{\SpecialStringTok}[1]{\textcolor[rgb]{0.73,0.40,0.53}{{#1}}}
    \newcommand{\ImportTok}[1]{{#1}}
    \newcommand{\DocumentationTok}[1]{\textcolor[rgb]{0.73,0.13,0.13}{\textit{{#1}}}}
    \newcommand{\AnnotationTok}[1]{\textcolor[rgb]{0.38,0.63,0.69}{\textbf{\textit{{#1}}}}}
    \newcommand{\CommentVarTok}[1]{\textcolor[rgb]{0.38,0.63,0.69}{\textbf{\textit{{#1}}}}}
    \newcommand{\VariableTok}[1]{\textcolor[rgb]{0.10,0.09,0.49}{{#1}}}
    \newcommand{\ControlFlowTok}[1]{\textcolor[rgb]{0.00,0.44,0.13}{\textbf{{#1}}}}
    \newcommand{\OperatorTok}[1]{\textcolor[rgb]{0.40,0.40,0.40}{{#1}}}
    \newcommand{\BuiltInTok}[1]{{#1}}
    \newcommand{\ExtensionTok}[1]{{#1}}
    \newcommand{\PreprocessorTok}[1]{\textcolor[rgb]{0.74,0.48,0.00}{{#1}}}
    \newcommand{\AttributeTok}[1]{\textcolor[rgb]{0.49,0.56,0.16}{{#1}}}
    \newcommand{\InformationTok}[1]{\textcolor[rgb]{0.38,0.63,0.69}{\textbf{\textit{{#1}}}}}
    \newcommand{\WarningTok}[1]{\textcolor[rgb]{0.38,0.63,0.69}{\textbf{\textit{{#1}}}}}
    
    
    % Define a nice break command that doesn't care if a line doesn't already
    % exist.
    \def\br{\hspace*{\fill} \\* }
    % Math Jax compatibility definitions
    \def\gt{>}
    \def\lt{<}
    \let\Oldtex\TeX
    \let\Oldlatex\LaTeX
    \renewcommand{\TeX}{\textrm{\Oldtex}}
    \renewcommand{\LaTeX}{\textrm{\Oldlatex}}
    % Document parameters
    % Document title
    \title{CS-562-Homework3}
    
    
    
    
    
% Pygments definitions
\makeatletter
\def\PY@reset{\let\PY@it=\relax \let\PY@bf=\relax%
    \let\PY@ul=\relax \let\PY@tc=\relax%
    \let\PY@bc=\relax \let\PY@ff=\relax}
\def\PY@tok#1{\csname PY@tok@#1\endcsname}
\def\PY@toks#1+{\ifx\relax#1\empty\else%
    \PY@tok{#1}\expandafter\PY@toks\fi}
\def\PY@do#1{\PY@bc{\PY@tc{\PY@ul{%
    \PY@it{\PY@bf{\PY@ff{#1}}}}}}}
\def\PY#1#2{\PY@reset\PY@toks#1+\relax+\PY@do{#2}}

\expandafter\def\csname PY@tok@w\endcsname{\def\PY@tc##1{\textcolor[rgb]{0.73,0.73,0.73}{##1}}}
\expandafter\def\csname PY@tok@c\endcsname{\let\PY@it=\textit\def\PY@tc##1{\textcolor[rgb]{0.25,0.50,0.50}{##1}}}
\expandafter\def\csname PY@tok@cp\endcsname{\def\PY@tc##1{\textcolor[rgb]{0.74,0.48,0.00}{##1}}}
\expandafter\def\csname PY@tok@k\endcsname{\let\PY@bf=\textbf\def\PY@tc##1{\textcolor[rgb]{0.00,0.50,0.00}{##1}}}
\expandafter\def\csname PY@tok@kp\endcsname{\def\PY@tc##1{\textcolor[rgb]{0.00,0.50,0.00}{##1}}}
\expandafter\def\csname PY@tok@kt\endcsname{\def\PY@tc##1{\textcolor[rgb]{0.69,0.00,0.25}{##1}}}
\expandafter\def\csname PY@tok@o\endcsname{\def\PY@tc##1{\textcolor[rgb]{0.40,0.40,0.40}{##1}}}
\expandafter\def\csname PY@tok@ow\endcsname{\let\PY@bf=\textbf\def\PY@tc##1{\textcolor[rgb]{0.67,0.13,1.00}{##1}}}
\expandafter\def\csname PY@tok@nb\endcsname{\def\PY@tc##1{\textcolor[rgb]{0.00,0.50,0.00}{##1}}}
\expandafter\def\csname PY@tok@nf\endcsname{\def\PY@tc##1{\textcolor[rgb]{0.00,0.00,1.00}{##1}}}
\expandafter\def\csname PY@tok@nc\endcsname{\let\PY@bf=\textbf\def\PY@tc##1{\textcolor[rgb]{0.00,0.00,1.00}{##1}}}
\expandafter\def\csname PY@tok@nn\endcsname{\let\PY@bf=\textbf\def\PY@tc##1{\textcolor[rgb]{0.00,0.00,1.00}{##1}}}
\expandafter\def\csname PY@tok@ne\endcsname{\let\PY@bf=\textbf\def\PY@tc##1{\textcolor[rgb]{0.82,0.25,0.23}{##1}}}
\expandafter\def\csname PY@tok@nv\endcsname{\def\PY@tc##1{\textcolor[rgb]{0.10,0.09,0.49}{##1}}}
\expandafter\def\csname PY@tok@no\endcsname{\def\PY@tc##1{\textcolor[rgb]{0.53,0.00,0.00}{##1}}}
\expandafter\def\csname PY@tok@nl\endcsname{\def\PY@tc##1{\textcolor[rgb]{0.63,0.63,0.00}{##1}}}
\expandafter\def\csname PY@tok@ni\endcsname{\let\PY@bf=\textbf\def\PY@tc##1{\textcolor[rgb]{0.60,0.60,0.60}{##1}}}
\expandafter\def\csname PY@tok@na\endcsname{\def\PY@tc##1{\textcolor[rgb]{0.49,0.56,0.16}{##1}}}
\expandafter\def\csname PY@tok@nt\endcsname{\let\PY@bf=\textbf\def\PY@tc##1{\textcolor[rgb]{0.00,0.50,0.00}{##1}}}
\expandafter\def\csname PY@tok@nd\endcsname{\def\PY@tc##1{\textcolor[rgb]{0.67,0.13,1.00}{##1}}}
\expandafter\def\csname PY@tok@s\endcsname{\def\PY@tc##1{\textcolor[rgb]{0.73,0.13,0.13}{##1}}}
\expandafter\def\csname PY@tok@sd\endcsname{\let\PY@it=\textit\def\PY@tc##1{\textcolor[rgb]{0.73,0.13,0.13}{##1}}}
\expandafter\def\csname PY@tok@si\endcsname{\let\PY@bf=\textbf\def\PY@tc##1{\textcolor[rgb]{0.73,0.40,0.53}{##1}}}
\expandafter\def\csname PY@tok@se\endcsname{\let\PY@bf=\textbf\def\PY@tc##1{\textcolor[rgb]{0.73,0.40,0.13}{##1}}}
\expandafter\def\csname PY@tok@sr\endcsname{\def\PY@tc##1{\textcolor[rgb]{0.73,0.40,0.53}{##1}}}
\expandafter\def\csname PY@tok@ss\endcsname{\def\PY@tc##1{\textcolor[rgb]{0.10,0.09,0.49}{##1}}}
\expandafter\def\csname PY@tok@sx\endcsname{\def\PY@tc##1{\textcolor[rgb]{0.00,0.50,0.00}{##1}}}
\expandafter\def\csname PY@tok@m\endcsname{\def\PY@tc##1{\textcolor[rgb]{0.40,0.40,0.40}{##1}}}
\expandafter\def\csname PY@tok@gh\endcsname{\let\PY@bf=\textbf\def\PY@tc##1{\textcolor[rgb]{0.00,0.00,0.50}{##1}}}
\expandafter\def\csname PY@tok@gu\endcsname{\let\PY@bf=\textbf\def\PY@tc##1{\textcolor[rgb]{0.50,0.00,0.50}{##1}}}
\expandafter\def\csname PY@tok@gd\endcsname{\def\PY@tc##1{\textcolor[rgb]{0.63,0.00,0.00}{##1}}}
\expandafter\def\csname PY@tok@gi\endcsname{\def\PY@tc##1{\textcolor[rgb]{0.00,0.63,0.00}{##1}}}
\expandafter\def\csname PY@tok@gr\endcsname{\def\PY@tc##1{\textcolor[rgb]{1.00,0.00,0.00}{##1}}}
\expandafter\def\csname PY@tok@ge\endcsname{\let\PY@it=\textit}
\expandafter\def\csname PY@tok@gs\endcsname{\let\PY@bf=\textbf}
\expandafter\def\csname PY@tok@gp\endcsname{\let\PY@bf=\textbf\def\PY@tc##1{\textcolor[rgb]{0.00,0.00,0.50}{##1}}}
\expandafter\def\csname PY@tok@go\endcsname{\def\PY@tc##1{\textcolor[rgb]{0.53,0.53,0.53}{##1}}}
\expandafter\def\csname PY@tok@gt\endcsname{\def\PY@tc##1{\textcolor[rgb]{0.00,0.27,0.87}{##1}}}
\expandafter\def\csname PY@tok@err\endcsname{\def\PY@bc##1{\setlength{\fboxsep}{0pt}\fcolorbox[rgb]{1.00,0.00,0.00}{1,1,1}{\strut ##1}}}
\expandafter\def\csname PY@tok@kc\endcsname{\let\PY@bf=\textbf\def\PY@tc##1{\textcolor[rgb]{0.00,0.50,0.00}{##1}}}
\expandafter\def\csname PY@tok@kd\endcsname{\let\PY@bf=\textbf\def\PY@tc##1{\textcolor[rgb]{0.00,0.50,0.00}{##1}}}
\expandafter\def\csname PY@tok@kn\endcsname{\let\PY@bf=\textbf\def\PY@tc##1{\textcolor[rgb]{0.00,0.50,0.00}{##1}}}
\expandafter\def\csname PY@tok@kr\endcsname{\let\PY@bf=\textbf\def\PY@tc##1{\textcolor[rgb]{0.00,0.50,0.00}{##1}}}
\expandafter\def\csname PY@tok@bp\endcsname{\def\PY@tc##1{\textcolor[rgb]{0.00,0.50,0.00}{##1}}}
\expandafter\def\csname PY@tok@fm\endcsname{\def\PY@tc##1{\textcolor[rgb]{0.00,0.00,1.00}{##1}}}
\expandafter\def\csname PY@tok@vc\endcsname{\def\PY@tc##1{\textcolor[rgb]{0.10,0.09,0.49}{##1}}}
\expandafter\def\csname PY@tok@vg\endcsname{\def\PY@tc##1{\textcolor[rgb]{0.10,0.09,0.49}{##1}}}
\expandafter\def\csname PY@tok@vi\endcsname{\def\PY@tc##1{\textcolor[rgb]{0.10,0.09,0.49}{##1}}}
\expandafter\def\csname PY@tok@vm\endcsname{\def\PY@tc##1{\textcolor[rgb]{0.10,0.09,0.49}{##1}}}
\expandafter\def\csname PY@tok@sa\endcsname{\def\PY@tc##1{\textcolor[rgb]{0.73,0.13,0.13}{##1}}}
\expandafter\def\csname PY@tok@sb\endcsname{\def\PY@tc##1{\textcolor[rgb]{0.73,0.13,0.13}{##1}}}
\expandafter\def\csname PY@tok@sc\endcsname{\def\PY@tc##1{\textcolor[rgb]{0.73,0.13,0.13}{##1}}}
\expandafter\def\csname PY@tok@dl\endcsname{\def\PY@tc##1{\textcolor[rgb]{0.73,0.13,0.13}{##1}}}
\expandafter\def\csname PY@tok@s2\endcsname{\def\PY@tc##1{\textcolor[rgb]{0.73,0.13,0.13}{##1}}}
\expandafter\def\csname PY@tok@sh\endcsname{\def\PY@tc##1{\textcolor[rgb]{0.73,0.13,0.13}{##1}}}
\expandafter\def\csname PY@tok@s1\endcsname{\def\PY@tc##1{\textcolor[rgb]{0.73,0.13,0.13}{##1}}}
\expandafter\def\csname PY@tok@mb\endcsname{\def\PY@tc##1{\textcolor[rgb]{0.40,0.40,0.40}{##1}}}
\expandafter\def\csname PY@tok@mf\endcsname{\def\PY@tc##1{\textcolor[rgb]{0.40,0.40,0.40}{##1}}}
\expandafter\def\csname PY@tok@mh\endcsname{\def\PY@tc##1{\textcolor[rgb]{0.40,0.40,0.40}{##1}}}
\expandafter\def\csname PY@tok@mi\endcsname{\def\PY@tc##1{\textcolor[rgb]{0.40,0.40,0.40}{##1}}}
\expandafter\def\csname PY@tok@il\endcsname{\def\PY@tc##1{\textcolor[rgb]{0.40,0.40,0.40}{##1}}}
\expandafter\def\csname PY@tok@mo\endcsname{\def\PY@tc##1{\textcolor[rgb]{0.40,0.40,0.40}{##1}}}
\expandafter\def\csname PY@tok@ch\endcsname{\let\PY@it=\textit\def\PY@tc##1{\textcolor[rgb]{0.25,0.50,0.50}{##1}}}
\expandafter\def\csname PY@tok@cm\endcsname{\let\PY@it=\textit\def\PY@tc##1{\textcolor[rgb]{0.25,0.50,0.50}{##1}}}
\expandafter\def\csname PY@tok@cpf\endcsname{\let\PY@it=\textit\def\PY@tc##1{\textcolor[rgb]{0.25,0.50,0.50}{##1}}}
\expandafter\def\csname PY@tok@c1\endcsname{\let\PY@it=\textit\def\PY@tc##1{\textcolor[rgb]{0.25,0.50,0.50}{##1}}}
\expandafter\def\csname PY@tok@cs\endcsname{\let\PY@it=\textit\def\PY@tc##1{\textcolor[rgb]{0.25,0.50,0.50}{##1}}}

\def\PYZbs{\char`\\}
\def\PYZus{\char`\_}
\def\PYZob{\char`\{}
\def\PYZcb{\char`\}}
\def\PYZca{\char`\^}
\def\PYZam{\char`\&}
\def\PYZlt{\char`\<}
\def\PYZgt{\char`\>}
\def\PYZsh{\char`\#}
\def\PYZpc{\char`\%}
\def\PYZdl{\char`\$}
\def\PYZhy{\char`\-}
\def\PYZsq{\char`\'}
\def\PYZdq{\char`\"}
\def\PYZti{\char`\~}
% for compatibility with earlier versions
\def\PYZat{@}
\def\PYZlb{[}
\def\PYZrb{]}
\makeatother


    % For linebreaks inside Verbatim environment from package fancyvrb. 
    \makeatletter
        \newbox\Wrappedcontinuationbox 
        \newbox\Wrappedvisiblespacebox 
        \newcommand*\Wrappedvisiblespace {\textcolor{red}{\textvisiblespace}} 
        \newcommand*\Wrappedcontinuationsymbol {\textcolor{red}{\llap{\tiny$\m@th\hookrightarrow$}}} 
        \newcommand*\Wrappedcontinuationindent {3ex } 
        \newcommand*\Wrappedafterbreak {\kern\Wrappedcontinuationindent\copy\Wrappedcontinuationbox} 
        % Take advantage of the already applied Pygments mark-up to insert 
        % potential linebreaks for TeX processing. 
        %        {, <, #, %, $, ' and ": go to next line. 
        %        _, }, ^, &, >, - and ~: stay at end of broken line. 
        % Use of \textquotesingle for straight quote. 
        \newcommand*\Wrappedbreaksatspecials {% 
            \def\PYGZus{\discretionary{\char`\_}{\Wrappedafterbreak}{\char`\_}}% 
            \def\PYGZob{\discretionary{}{\Wrappedafterbreak\char`\{}{\char`\{}}% 
            \def\PYGZcb{\discretionary{\char`\}}{\Wrappedafterbreak}{\char`\}}}% 
            \def\PYGZca{\discretionary{\char`\^}{\Wrappedafterbreak}{\char`\^}}% 
            \def\PYGZam{\discretionary{\char`\&}{\Wrappedafterbreak}{\char`\&}}% 
            \def\PYGZlt{\discretionary{}{\Wrappedafterbreak\char`\<}{\char`\<}}% 
            \def\PYGZgt{\discretionary{\char`\>}{\Wrappedafterbreak}{\char`\>}}% 
            \def\PYGZsh{\discretionary{}{\Wrappedafterbreak\char`\#}{\char`\#}}% 
            \def\PYGZpc{\discretionary{}{\Wrappedafterbreak\char`\%}{\char`\%}}% 
            \def\PYGZdl{\discretionary{}{\Wrappedafterbreak\char`\$}{\char`\$}}% 
            \def\PYGZhy{\discretionary{\char`\-}{\Wrappedafterbreak}{\char`\-}}% 
            \def\PYGZsq{\discretionary{}{\Wrappedafterbreak\textquotesingle}{\textquotesingle}}% 
            \def\PYGZdq{\discretionary{}{\Wrappedafterbreak\char`\"}{\char`\"}}% 
            \def\PYGZti{\discretionary{\char`\~}{\Wrappedafterbreak}{\char`\~}}% 
        } 
        % Some characters . , ; ? ! / are not pygmentized. 
        % This macro makes them "active" and they will insert potential linebreaks 
        \newcommand*\Wrappedbreaksatpunct {% 
            \lccode`\~`\.\lowercase{\def~}{\discretionary{\hbox{\char`\.}}{\Wrappedafterbreak}{\hbox{\char`\.}}}% 
            \lccode`\~`\,\lowercase{\def~}{\discretionary{\hbox{\char`\,}}{\Wrappedafterbreak}{\hbox{\char`\,}}}% 
            \lccode`\~`\;\lowercase{\def~}{\discretionary{\hbox{\char`\;}}{\Wrappedafterbreak}{\hbox{\char`\;}}}% 
            \lccode`\~`\:\lowercase{\def~}{\discretionary{\hbox{\char`\:}}{\Wrappedafterbreak}{\hbox{\char`\:}}}% 
            \lccode`\~`\?\lowercase{\def~}{\discretionary{\hbox{\char`\?}}{\Wrappedafterbreak}{\hbox{\char`\?}}}% 
            \lccode`\~`\!\lowercase{\def~}{\discretionary{\hbox{\char`\!}}{\Wrappedafterbreak}{\hbox{\char`\!}}}% 
            \lccode`\~`\/\lowercase{\def~}{\discretionary{\hbox{\char`\/}}{\Wrappedafterbreak}{\hbox{\char`\/}}}% 
            \catcode`\.\active
            \catcode`\,\active 
            \catcode`\;\active
            \catcode`\:\active
            \catcode`\?\active
            \catcode`\!\active
            \catcode`\/\active 
            \lccode`\~`\~ 	
        }
    \makeatother

    \let\OriginalVerbatim=\Verbatim
    \makeatletter
    \renewcommand{\Verbatim}[1][1]{%
        %\parskip\z@skip
        \sbox\Wrappedcontinuationbox {\Wrappedcontinuationsymbol}%
        \sbox\Wrappedvisiblespacebox {\FV@SetupFont\Wrappedvisiblespace}%
        \def\FancyVerbFormatLine ##1{\hsize\linewidth
            \vtop{\raggedright\hyphenpenalty\z@\exhyphenpenalty\z@
                \doublehyphendemerits\z@\finalhyphendemerits\z@
                \strut ##1\strut}%
        }%
        % If the linebreak is at a space, the latter will be displayed as visible
        % space at end of first line, and a continuation symbol starts next line.
        % Stretch/shrink are however usually zero for typewriter font.
        \def\FV@Space {%
            \nobreak\hskip\z@ plus\fontdimen3\font minus\fontdimen4\font
            \discretionary{\copy\Wrappedvisiblespacebox}{\Wrappedafterbreak}
            {\kern\fontdimen2\font}%
        }%
        
        % Allow breaks at special characters using \PYG... macros.
        \Wrappedbreaksatspecials
        % Breaks at punctuation characters . , ; ? ! and / need catcode=\active 	
        \OriginalVerbatim[#1,codes*=\Wrappedbreaksatpunct]%
    }
    \makeatother

    % Exact colors from NB
    \definecolor{incolor}{HTML}{303F9F}
    \definecolor{outcolor}{HTML}{D84315}
    \definecolor{cellborder}{HTML}{CFCFCF}
    \definecolor{cellbackground}{HTML}{F7F7F7}
    
    % prompt
    \makeatletter
    \newcommand{\boxspacing}{\kern\kvtcb@left@rule\kern\kvtcb@boxsep}
    \makeatother
    \newcommand{\prompt}[4]{
        \ttfamily\llap{{\color{#2}[#3]:\hspace{3pt}#4}}\vspace{-\baselineskip}
    }
    

    
    % Prevent overflowing lines due to hard-to-break entities
    \sloppy 
    % Setup hyperref package
    \hypersetup{
      breaklinks=true,  % so long urls are correctly broken across lines
      colorlinks=true,
      urlcolor=urlcolor,
      linkcolor=linkcolor,
      citecolor=citecolor,
      }
    % Slightly bigger margins than the latex defaults
    
    \geometry{verbose,tmargin=1in,bmargin=1in,lmargin=1in,rmargin=1in}
    
    

\begin{document}
    
    \maketitle
    
    

    
    \hypertarget{problem-1.-fastmap-20-points}{%
\section{Problem 1. (FastMap) {[}20
Points{]}}\label{problem-1.-fastmap-20-points}}

Using the \textbf{FastMap} algorithm to embed (map) the following
5-dimensional points into 3-dimensional points. Use the Euclidian
distance (L2 norm) as the distance between the points in the original
5-dimensional space. Show the 3-d points of the mapping for each point.

p1: (5.1, 3.5, 1.4, 0.2, 3.0)\\
p2: (4.9, 3.0, 1.4, 0.2, 4.0)\\
p3: (4.7, 3.2, 1.3, 0.2, 2.0)\\
p4: (6.9, 3.1, 4.9, 1.5, 2.0)\\
p5: (5.5, 2.3, 4.0, 1.3, 3.0)\\
p6: (6.5, 2.8, 4.6, 1.5, 5.0)\\
p7: (7.2, 3.0, 5.8, 1.6, 7.0)\\
p8: (7.4, 2.8, 6.1, 1.9, 8.0)

    \begin{tcolorbox}[breakable, size=fbox, boxrule=1pt, pad at break*=1mm,colback=cellbackground, colframe=cellborder]
\prompt{In}{incolor}{1}{\boxspacing}
\begin{Verbatim}[commandchars=\\\{\}]
\PY{k+kn}{import} \PY{n+nn}{numpy} \PY{k}{as} \PY{n+nn}{np}
\PY{k+kn}{import} \PY{n+nn}{pandas} \PY{k}{as} \PY{n+nn}{pd}
\PY{k+kn}{from} \PY{n+nn}{scipy}\PY{n+nn}{.}\PY{n+nn}{spatial} \PY{k}{import} \PY{n}{distance}
\PY{k+kn}{import} \PY{n+nn}{math}
\PY{k+kn}{import} \PY{n+nn}{matplotlib}\PY{n+nn}{.}\PY{n+nn}{pyplot} \PY{k}{as} \PY{n+nn}{plt}

\PY{c+c1}{\PYZsh{} global variables}
\PY{n}{points} \PY{o}{=} \PY{p}{[}\PY{p}{]}
\PY{n}{target\PYZus{}dimension} \PY{o}{=} \PY{l+m+mi}{3}
\PY{n}{dist\PYZus{}array} \PY{o}{=} \PY{n}{np}\PY{o}{.}\PY{n}{zeros}\PY{p}{(}\PY{p}{(}\PY{l+m+mi}{8}\PY{p}{,} \PY{l+m+mi}{8}\PY{p}{)}\PY{p}{,} \PY{n}{dtype} \PY{o}{=} \PY{n}{np}\PY{o}{.}\PY{n}{float}\PY{p}{)}
\PY{n}{mapped} \PY{o}{=} \PY{n}{np}\PY{o}{.}\PY{n}{zeros}\PY{p}{(}\PY{p}{(}\PY{l+m+mi}{8}\PY{p}{,} \PY{n}{target\PYZus{}dimension}\PY{p}{)}\PY{p}{,} \PY{n}{dtype} \PY{o}{=} \PY{n}{np}\PY{o}{.}\PY{n}{float}\PY{p}{)}
\PY{n}{pivots} \PY{o}{=} \PY{n}{np}\PY{o}{.}\PY{n}{zeros}\PY{p}{(}\PY{p}{(}\PY{l+m+mi}{2}\PY{p}{,} \PY{l+m+mi}{4}\PY{p}{)}\PY{p}{,} \PY{n}{dtype} \PY{o}{=} \PY{n}{np}\PY{o}{.}\PY{n}{int}\PY{p}{)}
\PY{n}{colNum} \PY{o}{=} \PY{l+m+mi}{0}


\PY{k}{def} \PY{n+nf}{preprocess}\PY{p}{(}\PY{p}{)}\PY{p}{:} 
    \PY{l+s+sd}{\PYZsq{}\PYZsq{}\PYZsq{}}
\PY{l+s+sd}{    Read data and append to a global matrix}
\PY{l+s+sd}{    \PYZsq{}\PYZsq{}\PYZsq{}}
    \PY{n}{O1} \PY{o}{=} \PY{p}{[}\PY{l+m+mf}{5.1}\PY{p}{,} \PY{l+m+mf}{3.5}\PY{p}{,} \PY{l+m+mf}{1.4}\PY{p}{,} \PY{l+m+mf}{0.2}\PY{p}{,} \PY{l+m+mf}{3.0}\PY{p}{]}
    \PY{n}{O2} \PY{o}{=} \PY{p}{[}\PY{l+m+mf}{4.9}\PY{p}{,} \PY{l+m+mf}{3.0}\PY{p}{,} \PY{l+m+mf}{1.4}\PY{p}{,} \PY{l+m+mf}{0.2}\PY{p}{,} \PY{l+m+mf}{4.0}\PY{p}{]}
    \PY{n}{O3} \PY{o}{=} \PY{p}{[}\PY{l+m+mf}{4.7}\PY{p}{,} \PY{l+m+mf}{3.2}\PY{p}{,} \PY{l+m+mf}{1.3}\PY{p}{,} \PY{l+m+mf}{0.2}\PY{p}{,} \PY{l+m+mf}{2.0}\PY{p}{]}
    \PY{n}{O4} \PY{o}{=} \PY{p}{[}\PY{l+m+mf}{6.9}\PY{p}{,} \PY{l+m+mf}{3.1}\PY{p}{,} \PY{l+m+mf}{4.9}\PY{p}{,} \PY{l+m+mf}{1.5}\PY{p}{,} \PY{l+m+mf}{2.0}\PY{p}{]}
    \PY{n}{O5} \PY{o}{=} \PY{p}{[}\PY{l+m+mf}{5.5}\PY{p}{,} \PY{l+m+mf}{2.3}\PY{p}{,} \PY{l+m+mf}{4.0}\PY{p}{,} \PY{l+m+mf}{1.3}\PY{p}{,} \PY{l+m+mf}{3.0}\PY{p}{]}
    \PY{n}{O6} \PY{o}{=} \PY{p}{[}\PY{l+m+mf}{6.5}\PY{p}{,} \PY{l+m+mf}{2.8}\PY{p}{,} \PY{l+m+mf}{4.6}\PY{p}{,} \PY{l+m+mf}{1.5}\PY{p}{,} \PY{l+m+mf}{5.0}\PY{p}{]}
    \PY{n}{O7} \PY{o}{=} \PY{p}{[}\PY{l+m+mf}{7.2}\PY{p}{,} \PY{l+m+mf}{3.0}\PY{p}{,} \PY{l+m+mf}{5.8}\PY{p}{,} \PY{l+m+mf}{1.6}\PY{p}{,} \PY{l+m+mf}{7.0}\PY{p}{]}
    \PY{n}{O8} \PY{o}{=} \PY{p}{[}\PY{l+m+mf}{7.4}\PY{p}{,} \PY{l+m+mf}{2.8}\PY{p}{,} \PY{l+m+mf}{6.1}\PY{p}{,} \PY{l+m+mf}{1.9}\PY{p}{,} \PY{l+m+mf}{8.0}\PY{p}{]}
    \PY{n}{points}\PY{o}{.}\PY{n}{append}\PY{p}{(}\PY{n}{O1}\PY{p}{)}
    \PY{n}{points}\PY{o}{.}\PY{n}{append}\PY{p}{(}\PY{n}{O2}\PY{p}{)}
    \PY{n}{points}\PY{o}{.}\PY{n}{append}\PY{p}{(}\PY{n}{O3}\PY{p}{)}
    \PY{n}{points}\PY{o}{.}\PY{n}{append}\PY{p}{(}\PY{n}{O4}\PY{p}{)}
    \PY{n}{points}\PY{o}{.}\PY{n}{append}\PY{p}{(}\PY{n}{O5}\PY{p}{)}
    \PY{n}{points}\PY{o}{.}\PY{n}{append}\PY{p}{(}\PY{n}{O6}\PY{p}{)}
    \PY{n}{points}\PY{o}{.}\PY{n}{append}\PY{p}{(}\PY{n}{O7}\PY{p}{)}
    \PY{n}{points}\PY{o}{.}\PY{n}{append}\PY{p}{(}\PY{n}{O8}\PY{p}{)}
    
    \PY{k}{return} \PY{k+kc}{None}


\PY{k}{def} \PY{n+nf}{dist}\PY{p}{(}\PY{n}{obj\PYZus{}a}\PY{p}{,} \PY{n}{obj\PYZus{}b}\PY{p}{)}\PY{p}{:}
    \PY{l+s+sd}{\PYZsq{}\PYZsq{}\PYZsq{}}
\PY{l+s+sd}{        return L2 distance}
\PY{l+s+sd}{    \PYZsq{}\PYZsq{}\PYZsq{}}
    \PY{n}{dis} \PY{o}{=} \PY{l+m+mi}{0}
    \PY{k}{for} \PY{n}{i} \PY{o+ow}{in} \PY{n+nb}{range}\PY{p}{(}\PY{n+nb}{len}\PY{p}{(}\PY{n}{obj\PYZus{}a}\PY{p}{)}\PY{p}{)}\PY{p}{:}
        \PY{n}{dis} \PY{o}{+}\PY{o}{=} \PY{p}{(}\PY{n}{obj\PYZus{}a}\PY{p}{[}\PY{n}{i}\PY{p}{]} \PY{o}{\PYZhy{}} \PY{n}{obj\PYZus{}b}\PY{p}{[}\PY{n}{i}\PY{p}{]}\PY{p}{)}\PY{o}{*}\PY{o}{*}\PY{l+m+mi}{2}
    \PY{k}{return} \PY{n}{math}\PY{o}{.}\PY{n}{sqrt}\PY{p}{(}\PY{n}{dis}\PY{p}{)}


\PY{k}{def} \PY{n+nf}{init\PYZus{}distances}\PY{p}{(}\PY{p}{)}\PY{p}{:}
    \PY{l+s+sd}{\PYZsq{}\PYZsq{}\PYZsq{}}
\PY{l+s+sd}{        create distance matrix: dist\PYZus{}array[N][N]}
\PY{l+s+sd}{        dist\PYZus{}array[i][j] is the distance between object i and j in original space}
\PY{l+s+sd}{    \PYZsq{}\PYZsq{}\PYZsq{}}
    \PY{k}{for} \PY{n}{x} \PY{o+ow}{in} \PY{n+nb}{range}\PY{p}{(}\PY{l+m+mi}{0}\PY{p}{,} \PY{l+m+mi}{8}\PY{p}{)}\PY{p}{:}
        \PY{k}{for} \PY{n}{y} \PY{o+ow}{in} \PY{n+nb}{range}\PY{p}{(}\PY{l+m+mi}{0}\PY{p}{,} \PY{l+m+mi}{8}\PY{p}{)}\PY{p}{:} 
            \PY{n}{dist\PYZus{}array}\PY{p}{[}\PY{n}{x}\PY{p}{]}\PY{p}{[}\PY{n}{y}\PY{p}{]} \PY{o}{=} \PY{n}{dist}\PY{p}{(}\PY{n}{points}\PY{p}{[}\PY{n}{x}\PY{p}{]}\PY{p}{,} \PY{n}{points}\PY{p}{[}\PY{n}{y}\PY{p}{]}\PY{p}{)}
    
    \PY{k}{return} \PY{k+kc}{None}


\PY{k}{def} \PY{n+nf}{choose\PYZus{}dist\PYZus{}obj}\PY{p}{(}\PY{n}{distances}\PY{p}{)}\PY{p}{:}
    \PY{l+s+sd}{\PYZsq{}\PYZsq{}\PYZsq{}}
\PY{l+s+sd}{        find two points whose distance is the longest}
\PY{l+s+sd}{    \PYZsq{}\PYZsq{}\PYZsq{}}
    \PY{c+c1}{\PYZsh{} randomly choose object b}
    \PY{n}{obj\PYZus{}b} \PY{o}{=} \PY{n}{np}\PY{o}{.}\PY{n}{random}\PY{o}{.}\PY{n}{randint}\PY{p}{(}\PY{l+m+mi}{0}\PY{p}{,}\PY{l+m+mi}{7}\PY{p}{)}

    \PY{c+c1}{\PYZsh{} Iterate until convergence}
    \PY{k}{while} \PY{k+kc}{True}\PY{p}{:} 
        \PY{c+c1}{\PYZsh{} use the distance matrix }

        \PY{c+c1}{\PYZsh{} find the farthest object from object b}
        \PY{n}{farthest} \PY{o}{=} \PY{n+nb}{max}\PY{p}{(}\PY{n}{distances}\PY{p}{[}\PY{n}{obj\PYZus{}b}\PY{p}{]}\PY{p}{)}
        \PY{n}{obj\PYZus{}a} \PY{o}{=} \PY{n}{distances}\PY{p}{[}\PY{n}{obj\PYZus{}b}\PY{p}{]}\PY{o}{.}\PY{n}{tolist}\PY{p}{(}\PY{p}{)}\PY{o}{.}\PY{n}{index}\PY{p}{(}\PY{n}{farthest}\PY{p}{)}

        \PY{c+c1}{\PYZsh{} find the farthest object from object a}
        \PY{n}{tmp} \PY{o}{=} \PY{n+nb}{max}\PY{p}{(}\PY{n}{distances}\PY{p}{[}\PY{n}{obj\PYZus{}a}\PY{p}{]}\PY{p}{)}
        \PY{n}{tmpObj} \PY{o}{=} \PY{n}{distances}\PY{p}{[}\PY{n}{obj\PYZus{}a}\PY{p}{]}\PY{o}{.}\PY{n}{tolist}\PY{p}{(}\PY{p}{)}\PY{o}{.}\PY{n}{index}\PY{p}{(}\PY{n}{tmp}\PY{p}{)}

        \PY{c+c1}{\PYZsh{} when the farthest object of object a is exatly object b,}
        \PY{c+c1}{\PYZsh{} it converges}
        \PY{k}{if} \PY{p}{(}\PY{n}{tmpObj} \PY{o}{==} \PY{n}{obj\PYZus{}b}\PY{p}{)}\PY{p}{:}
            \PY{k}{break}
        \PY{k}{else}\PY{p}{:}
            \PY{n}{obj\PYZus{}b} \PY{o}{=} \PY{n}{tmpObj}

    \PY{c+c1}{\PYZsh{} return a smaller object id}
    \PY{k}{if} \PY{n}{obj\PYZus{}a} \PY{o}{\PYZlt{}} \PY{n}{obj\PYZus{}b}\PY{p}{:}
        \PY{k}{return} \PY{p}{(}\PY{n}{obj\PYZus{}a}\PY{p}{,} \PY{n}{obj\PYZus{}b}\PY{p}{)}
    \PY{k}{else}\PY{p}{:}
        \PY{k}{return} \PY{p}{(}\PY{n}{obj\PYZus{}b}\PY{p}{,} \PY{n}{obj\PYZus{}a}\PY{p}{)}


\PY{k}{def} \PY{n+nf}{fastMap}\PY{p}{(}\PY{n}{k}\PY{p}{,} \PY{n}{distances}\PY{p}{)}\PY{p}{:}
    \PY{l+s+sd}{\PYZsq{}\PYZsq{}\PYZsq{}}
\PY{l+s+sd}{        k: target dimension}
\PY{l+s+sd}{        distances: distance matrix or last projection}
\PY{l+s+sd}{        This function will be called recursively.}
\PY{l+s+sd}{    \PYZsq{}\PYZsq{}\PYZsq{}}
    \PY{k}{global} \PY{n}{colNum}

    \PY{c+c1}{\PYZsh{} recursion exit point}
    \PY{k}{if} \PY{n}{k} \PY{o}{\PYZlt{}}\PY{o}{=} \PY{l+m+mi}{0}\PY{p}{:}
        \PY{k}{return} \PY{k+kc}{None}
    \PY{k}{else}\PY{p}{:}
        \PY{k}{pass}
        \PY{c+c1}{\PYZsh{} colNum += 1}
    
    \PY{c+c1}{\PYZsh{} choose pivot object}
    \PY{n}{pivots} \PY{o}{=} \PY{n}{choose\PYZus{}dist\PYZus{}obj}\PY{p}{(}\PY{n}{distances}\PY{p}{)}
    
    \PY{n}{a} \PY{o}{=} \PY{n}{pivots}\PY{p}{[}\PY{l+m+mi}{0}\PY{p}{]}   \PY{c+c1}{\PYZsh{} pivot a}
    \PY{n}{b} \PY{o}{=} \PY{n}{pivots}\PY{p}{[}\PY{l+m+mi}{1}\PY{p}{]}   \PY{c+c1}{\PYZsh{} pivot b}

    \PY{n}{farthest} \PY{o}{=} \PY{n}{distances}\PY{p}{[}\PY{n}{a}\PY{p}{]}\PY{p}{[}\PY{n}{b}\PY{p}{]}
    \PY{k}{if} \PY{n}{farthest} \PY{o}{==} \PY{l+m+mi}{0}\PY{p}{:}
        \PY{k}{for} \PY{n}{i} \PY{o+ow}{in} \PY{n+nb}{range}\PY{p}{(}\PY{n+nb}{len}\PY{p}{(}\PY{n}{points}\PY{p}{)}\PY{p}{)}\PY{p}{:}
            \PY{n}{mapped}\PY{p}{[}\PY{n}{i}\PY{p}{]}\PY{p}{[}\PY{n}{colNum}\PY{p}{]} \PY{o}{=} \PY{l+m+mi}{0}

    \PY{c+c1}{\PYZsh{} project objects on line(Oa, Ob)}
    \PY{k}{for} \PY{n}{i} \PY{o+ow}{in} \PY{n+nb}{range}\PY{p}{(}\PY{l+m+mi}{0}\PY{p}{,} \PY{n+nb}{len}\PY{p}{(}\PY{n}{points}\PY{p}{)}\PY{p}{)}\PY{p}{:}
        \PY{n}{temp} \PY{o}{=} \PY{l+m+mi}{0}
        \PY{k}{if} \PY{n}{i} \PY{o}{==} \PY{n}{a}\PY{p}{:}
            \PY{n}{mapped}\PY{p}{[}\PY{n}{i}\PY{p}{]}\PY{p}{[}\PY{n}{colNum}\PY{p}{]} \PY{o}{=} \PY{l+m+mi}{0}
        \PY{k}{elif} \PY{n}{i} \PY{o}{==} \PY{n}{b}\PY{p}{:}
            \PY{n}{mapped}\PY{p}{[}\PY{n}{i}\PY{p}{]}\PY{p}{[}\PY{n}{colNum}\PY{p}{]} \PY{o}{=} \PY{n}{farthest}
        \PY{k}{else}\PY{p}{:}
            \PY{c+c1}{\PYZsh{} cosine law}
            \PY{n}{temp} \PY{o}{=} \PY{p}{(}\PY{p}{(}\PY{n}{distances}\PY{p}{[}\PY{n}{a}\PY{p}{]}\PY{p}{[}\PY{n}{i}\PY{p}{]}\PY{o}{*}\PY{o}{*}\PY{l+m+mi}{2}\PY{p}{)} \PY{o}{+} \PY{p}{(}\PY{n}{farthest}\PY{o}{*}\PY{o}{*}\PY{l+m+mi}{2}\PY{p}{)} \PY{o}{\PYZhy{}} \PY{p}{(}\PY{n}{distances}\PY{p}{[}\PY{n}{b}\PY{p}{]}\PY{p}{[}\PY{n}{i}\PY{p}{]}\PY{o}{*}\PY{o}{*}\PY{l+m+mi}{2}\PY{p}{)}\PY{p}{)}\PY{o}{/}\PY{p}{(}\PY{l+m+mi}{2} \PY{o}{*} \PY{n}{farthest}\PY{p}{)}
            \PY{n}{mapped}\PY{p}{[}\PY{n}{i}\PY{p}{]}\PY{p}{[}\PY{n}{colNum}\PY{p}{]} \PY{o}{=} \PY{n}{temp}

    \PY{c+c1}{\PYZsh{} update distance matrix}
    \PY{c+c1}{\PYZsh{} projection will be the new distance matrix after dimension reduction}
    \PY{n}{projection} \PY{o}{=} \PY{n}{np}\PY{o}{.}\PY{n}{zeros}\PY{p}{(}\PY{p}{(}\PY{l+m+mi}{8}\PY{p}{,} \PY{l+m+mi}{8}\PY{p}{)}\PY{p}{)}
    \PY{k}{for} \PY{n}{i} \PY{o+ow}{in} \PY{n+nb}{range} \PY{p}{(}\PY{l+m+mi}{8}\PY{p}{)}\PY{p}{:}
        \PY{k}{for} \PY{n}{j} \PY{o+ow}{in} \PY{n+nb}{range} \PY{p}{(}\PY{l+m+mi}{8}\PY{p}{)}\PY{p}{:}
            \PY{c+c1}{\PYZsh{} dimensional reduction}
            \PY{n}{tmp} \PY{o}{=} \PY{p}{(}\PY{n}{distances}\PY{p}{[}\PY{n}{i}\PY{p}{]}\PY{p}{[}\PY{n}{j}\PY{p}{]} \PY{o}{*}\PY{o}{*} \PY{l+m+mi}{2}\PY{p}{)} \PY{o}{\PYZhy{}} \PY{p}{(}\PY{p}{(}\PY{n}{mapped}\PY{p}{[}\PY{n}{i}\PY{p}{]}\PY{p}{[}\PY{n}{colNum}\PY{p}{]} \PY{o}{\PYZhy{}} \PY{n}{mapped}\PY{p}{[}\PY{n}{j}\PY{p}{]}\PY{p}{[}\PY{n}{colNum}\PY{p}{]}\PY{p}{)} \PY{o}{*}\PY{o}{*} \PY{l+m+mi}{2}\PY{p}{)}
            \PY{n}{projection}\PY{p}{[}\PY{n}{i}\PY{p}{]}\PY{p}{[}\PY{n}{j}\PY{p}{]} \PY{o}{=} \PY{n}{np}\PY{o}{.}\PY{n}{sqrt}\PY{p}{(}\PY{n}{np}\PY{o}{.}\PY{n}{absolute}\PY{p}{(}\PY{n}{tmp}\PY{p}{)}\PY{p}{)}

    \PY{n}{colNum} \PY{o}{+}\PY{o}{=} \PY{l+m+mi}{1}

    \PY{c+c1}{\PYZsh{} recursion}
    \PY{n}{fastMap}\PY{p}{(}\PY{n}{k}\PY{o}{\PYZhy{}}\PY{l+m+mi}{1}\PY{p}{,} \PY{n}{projection}\PY{p}{)}

    \PY{k}{return} \PY{k+kc}{None}


\PY{c+c1}{\PYZsh{} stress function}
\PY{k}{def} \PY{n+nf}{get\PYZus{}stress}\PY{p}{(}\PY{n}{original\PYZus{}space\PYZus{}data}\PY{p}{,} \PY{n}{target\PYZus{}space\PYZus{}data}\PY{p}{)}\PY{p}{:}
    \PY{l+s+sd}{\PYZsq{}\PYZsq{}\PYZsq{}}
\PY{l+s+sd}{        if the dimensional reduction maintains the dissimilarity between objects,}
\PY{l+s+sd}{        this return value would be low.}
\PY{l+s+sd}{    \PYZsq{}\PYZsq{}\PYZsq{}}
    \PY{n}{denominator} \PY{o}{=} \PY{l+m+mi}{0}
    \PY{n}{numerator} \PY{o}{=} \PY{l+m+mi}{0}
    \PY{k}{for} \PY{n}{i} \PY{o+ow}{in} \PY{n+nb}{range}\PY{p}{(}\PY{l+m+mi}{0}\PY{p}{,} \PY{n+nb}{len}\PY{p}{(}\PY{n}{original\PYZus{}space\PYZus{}data}\PY{p}{)}\PY{o}{\PYZhy{}}\PY{l+m+mi}{1}\PY{p}{)}\PY{p}{:}
        \PY{k}{for} \PY{n}{j} \PY{o+ow}{in} \PY{n+nb}{range}\PY{p}{(}\PY{n}{i}\PY{o}{+}\PY{l+m+mi}{1}\PY{p}{,} \PY{n+nb}{len}\PY{p}{(}\PY{n}{original\PYZus{}space\PYZus{}data}\PY{p}{)}\PY{p}{)}\PY{p}{:}
            \PY{n}{temp} \PY{o}{=} \PY{n}{dist}\PY{p}{(}\PY{n}{original\PYZus{}space\PYZus{}data}\PY{p}{[}\PY{n}{i}\PY{p}{]}\PY{p}{,} \PY{n}{original\PYZus{}space\PYZus{}data}\PY{p}{[}\PY{n}{j}\PY{p}{]}\PY{p}{)}
            \PY{n}{denominator} \PY{o}{+}\PY{o}{=} \PY{n}{temp}\PY{o}{*}\PY{o}{*}\PY{l+m+mi}{2}
            \PY{n}{numerator} \PY{o}{+}\PY{o}{=} \PY{p}{(}\PY{n}{temp} \PY{o}{\PYZhy{}} \PY{n}{dist}\PY{p}{(}\PY{n}{target\PYZus{}space\PYZus{}data}\PY{p}{[}\PY{n}{i}\PY{p}{]}\PY{p}{,} \PY{n}{target\PYZus{}space\PYZus{}data}\PY{p}{[}\PY{n}{j}\PY{p}{]}\PY{p}{)}\PY{p}{)}\PY{o}{*}\PY{o}{*}\PY{l+m+mi}{2}
    \PY{k}{return} \PY{n}{numerator}\PY{o}{/}\PY{n}{denominator}


\PY{k}{def} \PY{n+nf}{main}\PY{p}{(}\PY{p}{)}\PY{p}{:}
    \PY{n}{preprocess}\PY{p}{(}\PY{p}{)}
    \PY{n}{data} \PY{o}{=} \PY{n}{pd}\PY{o}{.}\PY{n}{DataFrame}\PY{p}{(}\PY{n}{np}\PY{o}{.}\PY{n}{array}\PY{p}{(}\PY{n}{points}\PY{p}{)}\PY{p}{)}
    \PY{n}{init\PYZus{}distances}\PY{p}{(}\PY{p}{)}
    \PY{n}{fastMap}\PY{p}{(}\PY{n}{target\PYZus{}dimension}\PY{p}{,} \PY{n}{dist\PYZus{}array}\PY{p}{)}
    \PY{n+nb}{print}\PY{p}{(}\PY{l+s+s1}{\PYZsq{}}\PY{l+s+s1}{Answer:}\PY{l+s+s1}{\PYZsq{}}\PY{p}{)}
    \PY{n+nb}{print}\PY{p}{(}\PY{n}{mapped}\PY{p}{)}
    \PY{n+nb}{print}\PY{p}{(}\PY{p}{)}
    \PY{n+nb}{print}\PY{p}{(}\PY{l+s+s1}{\PYZsq{}}\PY{l+s+s1}{stress function: }\PY{l+s+s1}{\PYZsq{}}\PY{p}{,} \PY{n}{get\PYZus{}stress}\PY{p}{(}\PY{n}{points}\PY{p}{,} \PY{n}{mapped}\PY{p}{)}\PY{p}{)}
    \PY{k}{return} \PY{k+kc}{None}


\PY{n}{main}\PY{p}{(}\PY{p}{)}
\end{Verbatim}
\end{tcolorbox}

    \begin{Verbatim}[commandchars=\\\{\}]
Answer:
[[0.89321455 0.77847574 0.        ]
 [1.57272992 0.         0.44752492]
 [0.         1.26154386 0.47790845]
 [3.05781916 4.44347949 0.44752492]
 [2.80330103 2.66288887 1.53966125]
 [4.93068837 2.04414852 0.68393317]
 [7.30058826 1.64592167 0.35510119]
 [8.32946577 1.26154386 0.47790845]]

stress function:  1.7385764793240135e-06
    \end{Verbatim}

    \hypertarget{problem-3-mapreduce-and-databases-30-points}{%
\section{Problem 3 (MapReduce and databases) {[}30
Points{]}}\label{problem-3-mapreduce-and-databases-30-points}}

\begin{enumerate}
\def\labelenumi{\arabic{enumi}.}
\tightlist
\item
  In the form of relational algebra implemented in SQL, relations are
  not sets, but bags; that is, tuples are allowed to appear more than
  once. There are extended definitions of union, intersection, and
  difference for bags, which we shall define below. Write MapReduce
  algorithms for computing the following operations on bags R and S:

  \begin{enumerate}
  \def\labelenumii{\arabic{enumii}.}
  \tightlist
  \item
    Bag Union, defined to be the bag of tuples in which tuple t appears
    the sum of the numbers of times it appears in R and S.
  \item
    Bag Intersection, defined to be the bag of tuples in which tuple t
    appears the minimum of the numbers of times it appears in R and S.
  \item
    Bag Difference, defined to be the bag of tuples in which the number
    of times a tuple t appears is equal to the number of times it
    appears in R minus the number of times it appears in S. A tuple that
    appears more times in S than in R does not appear in the difference.
  \end{enumerate}
\end{enumerate}

    \hypertarget{answer}{%
\subsubsection{Answer:}\label{answer}}

    \hypertarget{union}{%
\subsubsection{Union}\label{union}}

    \begin{verbatim}
Mapper(key null, Bag bag):
    // bag is R or S.
    for each tuple t in bag:
        emit(t, bag)

Reducer(tutple t, Iterator bags):
    for each bag in bags:
        emit(t,t)
\end{verbatim}

    \hypertarget{intersection}{%
\subsubsection{Intersection}\label{intersection}}

    \begin{verbatim}
Mapper(key null, Bag bag):
    for each tuple t in bag:
        emit(t, bag)         // it would be (ti, R) or (ti, S)

Reducer(tuple t, Iterator bags):
    for each bag in bags:
        emit((t, bag), 1)    // it would be ((ti, R), 1) or ((ti, S), 1)
        
Mapper((tuple t, Bag bag), Iterator values):
    // it would receive (ti, R), [1, 1, ... , 1]; (ti, S), [1, 1, ... , 1]
    emit(t, (bag, sum(values)))

Reducer(tuple t, Iterator (Bag bag, int count)):
    // count is the number of t's appearance times in different bags
    for i in range(min(count)):
        emit(t,t)
\end{verbatim}

    \hypertarget{difference}{%
\subsubsection{Difference}\label{difference}}

    \begin{verbatim}
Mapper(key null, Bag bag):
    for each tuple t in bag:
        emit(t, bag)
        
Reducer(tuple t, Iterator bags):
    for each bag in bags:
        emit((t, bag), 1)    // it would be ((ti, R), 1) or ((ti, S), 1)
        
Mapper((tuple t, Bag bag), Iterator values):
    // it would receive (ti, R), [1, 1, ... , 1]; (ti, S), [1, 1, ... , 1]
    emit(t, (bag, sum(values)))

Reducer(tuple t, Iterator (Bag bag, int count)):
    // count is the number of t's appearance times in different bags
    if the list is [R,count]:
        for i in range(count):
            emit(t,t)
    else if the list is [(R,count_R), (S,count_S)]:
        if count_R > count_S:
            for i in range(count_R-count_S):
                emit(t,t)
    else if the list is [(S,count_S), (R,count_R)]:
        if count_R > count_S:
            for i in range(count_R-count_S):
                emit(t,t)
    else:
        emit nothing
    
\end{verbatim}

    \begin{enumerate}
\def\labelenumi{\arabic{enumi}.}
\setcounter{enumi}{1}
\tightlist
\item
  The relational-algebra operation \[R(A, B) \bowtie_{B<C} S(C, D)\]
  produces all tuples (a, b, c, d) such that tuple (a, b) is in relation
  R, tuple (c, d) is in S, and b \textless{} c. Give a MapReduce
  implementation of this operation, assuming R and S are sets.
\end{enumerate}

    \hypertarget{answer}{%
\subsubsection{Answer:}\label{answer}}

    \begin{verbatim}
Mapper(key null, (Set R, Set S)):
    for each tuple (a,b) in R:
        emit(1,(a,(b,R)))
    for each tuple (c,d) in S:
        emit(1,(d,(c,S)))

Reducer(key, list (record1, (record2, Set))):
    // record1 would be a or d.
    // record2 would be b or c.
    // Set would be R or S.
    for each pair in list: 
        if pair.value.Set == R:
            emit((pair.record1, pair.value.record2), list)
            // this would emit((a,b),list)
            
Mapper(key (a,b), list (record1, (record2, Set))):
    for each pair in list:
        if (pair.value.Set == S) and (b > pair.value.record2):
            emit((a,b),(pair.value.record2,pair.record1))
            // this would emit((a,b),(c,d))
            
Reducer(key (a,b), list (c,d)):
    for each pair in list:
        emit((a,b,c,d),(a,b,c,d))
\end{verbatim}


    % Add a bibliography block to the postdoc
    
    
    
\end{document}
